


\newcommand{\besselI}[2]{\operatorname{I}_{#1}\!\left(#2\right)\!}
\renewcommand{\exp}[1]{\operatorname{e}^{#1}}

\newcommand{\sech}[1]{\operatorname{sech}\!\left(#1\right)}

\newcommand{\RealNumbers}[0]{\mathbb{R}}
\newcommand{\ComplexNumbers}[0]{\mathbb{C}}

\newcommand{\E}[2][]{%
\mathbb{E}_{#1}\!\left\{{#2}\right\}
}



\renewcommand{\Im}[1]{\mathfrak{I}\{#1\}}
\renewcommand{\Re}[1]{\mathfrak{R}\{#1\}}

\newcommand{\MI}[1]{\mathbb{I}(#1)}
\newcommand{\C}{\mathbb{C}}

\newcommand{\MIstart}[1]{\mathsf{I}(#1)}

\newcommand{\Istar}[1]{\mathsf{I}(#1)}
\newcommand{\Cstar}{\mathsf{C}}
\newcommand{\achievRate}[1]{\mathcal{R}_\mathrm{#1}}

\newcommand{\Bern}[1]{\operatorname{Bern}(#1)}

\newcommand{\entropy}[1]{\mathbb{H}(#1)}

\newcommand{\relEntropy}[2]{\mathcal{D}(#1||#2)}

\renewcommand{\Pr}[1]{\mathrm{Pr}\!\left[#1\right]}

\newcommand{\transpose}[1]{#1^\mathsf{T}}

\newcommand{\T}[1]{T\left(#1\right)}

\newcommand*\accaplong[1]{%
  \begingroup
    \acsetup{long-format=\titlecap}
    #1
  \endgroup  
}



\newcommand{\includetikz}[1]{%
    %\tikzsetnextfilename{./fig/tmp//#1}%
    \input{./fig/#1}%
}



